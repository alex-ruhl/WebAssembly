\chapter{Einführung}

Für sehr lange Zeit war JavaScript die einzige Möglichkeit für Interaktive Anwendungen im Browser. Rechenintensive Aufgaben, wie z. B. Spiele, wurden von der schlechten Perfomance JavaScripts zurückgehalten. Trotz der unaufhöhrlichen Optimierung von JavaScript-Engines genügt die Performance nicht immer den hohen Anforderungen für das Web \cite{mozilla17}. Ein erster Versuch dieses Problem anzugehen war asm.js. In der Theorie konnte mit dieser Technologie nahezu native Performance erzielt werden \cite{hamoz13}, jedoch war asm.js nie in allen Browsern konsistent schnell \cite{hamoz17}. Der Hauptgrund hierfür war, dass asm.js kein offizieller Web-Standard war. Bei WebAssembly ist dies anders, es ist seit 2017 offiziel von allen großen Browsern (Chrome, Edge, Firefox und Webkit)  unterstützt \cite{w3c17} und seit 2019 offizieler Standard für das Web \cite{w3c19}. Weitere Gründe warum WebAssembly schneller als asm.js ist werden unter 3.2 im Detail erläutert.

WebAssembly ist ein Binärinstruktionsformat für eine stackbasierte virtuelle Maschine. Es ist designed um als Kompilierziel von low-level Programmiersprachen wie C/C++/Rust zu dienen, hierdurch wird die Auslieferung im Web für Client- und Server-Anwendungen ermöglicht \cite{webasm20}.
Die Nutzungsmöglichkeiten beschränken sich nicht nur ausschließlich auf das Web, jedoch liegt das Hauptaugenmerk auf diesem Gebiet. Das Ziel von WebAssembly ist Clientseitige Applikationen, auf verschiedenen Plattformen, in der Ausführungszeit so nah wie möglich an native Applikationen zu bringen und dabei effizient zu sein. Außerdem bietet WebAssembly neben der low-level Assemblersprache ein Menschenlesbares Textformat (WAT - WebAssembly Text Format) um Entwicklern zu helfen den Code zu lesen und die Fehlersuche zu vereinfachen. Dieses Textformat kann auch genutzt werden um Code zu schreiben und diesen in das Binärformat zu kompilieren. 

Mit WebAssembly wird nicht versucht JavaScript zu ersetzen. JavaScript ist und bleibt für die meisten Anwendungsfälle die dominante Sprache im Web. WebAssembly und JavaScript sollen jedoch auf verschiedene Weise zusammen arbeiten. 
Zum Beispiel können Internetseiten wie üblich aufgebaut bleiben aber um schnelle WebAssembly-Module ergänzt werden. Diese Module können dann rechenintensive Aufgaben wie, Simulationen, Bild-/Ton-/Video-Verarbeitung, Visualisierung, Animationen, Kompression und Verschlüsselung übernehmen \cite{webasmfaq}.

Ziel dieser Arbeit ist es, WebAssembly mit allen Vor- und Nachteilen zu untersuchen und dem Leser einen Einblick in diese noch neue Technologie zu geben.